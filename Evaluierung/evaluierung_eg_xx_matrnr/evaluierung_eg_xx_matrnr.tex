%%Berichtvorlage für EDBV Teil 2 Evaluierung - WS 2015/2016

\documentclass[deutsch, paper=a4]{scrartcl}
\usepackage[ngerman]{babel}
\usepackage[utf8]{inputenc}
\usepackage{algorithmic}
\usepackage{algorithm}
\usepackage{graphicx}
\usepackage{amsmath,amssymb}
\usepackage{subcaption}
\captionsetup{compatibility=false}
\usepackage{multirow}
\usepackage{color}
\usepackage{enumitem}
\begin{document}


%%------------------------------------------------------
%% Ab hier tragt ihr eure Daten und Ergebnisse ein:
%%------------------------------------------------------

\title{Evaluierung: Projekttitel} %%Projekttitel des evaluierten Projekts hier eintragen

\subtitle{EDBV WS 2015/2016: EG\_XX} %%statt XX Evaluierungsgruppenbezeichnung der evaluierten Gruppe hier eintragen (zB.: 01)


%%Namen und Matrikelnummern hier eintragen
\author{Vorname Nachname (Matrikelnummer)}



%%------------------------------------------------------

\maketitle


%%------------------------------------------------------
\section{Bericht}
\textit{Bewertung Bericht Gesamteindruck: 0-8 Punkte\\ 
Begründung: }
\subsection{Gewählte Problemstellung}
\textit{Bewertung: 0-2 Punkte}\\
\begin{enumerate}[label=\alph*)]
\item \textit{0-1 Punkt | Begründung: }
\item \textit{0-1 Punkt | Begründung: }
\end{enumerate}

\subsection{Arbeitsteilung}
\textit{Bewertung: 0-2 Punkte}\\
\begin{enumerate}[label=\alph*)]
\item \textit{0-1 Punkt | Begründung: }
\item \textit{0-1 Punkt | Begründung: }
\end{enumerate}

\subsection{Methodik}
\textit{Bewertung: 0-8 Punkte}\\
\begin{enumerate}[label=\alph*)]
\item \textit{0-1 Punkt | Begründung: }
\item \textit{0-2 Punkte | Begründung: }
\item \textit{0-2 Punkte | Begründung: }
\item \textit{0-2 Punkte | Begründung: }
\item \textit{0-1 Punkt | Begründung: }
\end{enumerate}

\subsection{Implementierung}
\textit{Bewertung: 0-6 Punkte}\\
\begin{enumerate}[label=\alph*)]
\item \textit{0-2 Punkte | Begründung: }
\item \textit{0-2 Punkte | Begründung: }
\item \textit{0-2 Punkte | Begründung: }
\end{enumerate}

\subsection{Selbstevaluierung durch die Arbeitsgruppe}
\textit{Bewertung: 0-8 Punkte}\\
\begin{enumerate}[label=\alph*)]
\item \textit{0-2 Punkte | Begründung: }
\item \textit{0-2 Punkte | Begründung: }
\item \textit{0-2 Punkte | Begründung: }
\item \textit{0-2 Punkte | Begründung: }
\end{enumerate}

\subsection{Schlusswort}
\textit{Bewertung: 0-2 Punkte}\\
\begin{enumerate}[label=\alph*)]
\item \textit{0-1 Punkt | Begründung: }
\item \textit{0-1 Punkt | Begründung: }
\end{enumerate}

\subsection{Literatur}
\textit{Bewertung: 0-4 Punkte}\\
\begin{enumerate}[label=\alph*)]
\item \textit{0-1 Punkt | Begründung: }
\item \textit{0-1 Punkt | Begründung: }
\item \textit{0-1 Punkt | Begründung: }
\item \textit{0-1 Punkt | Begründung: }
\end{enumerate}
%%------------------------------------------------------

%%------------------------------------------------------
\section{Implementierung}
\textit{Bewertung Implementierung Gesamteindruck: 0-20 Punkte\\ 
Begründung: }
\subsection{Ausführbarkeit und Laufzeit}
\textit{Bewertung: 0-15 Punkte}\\
\begin{enumerate}[label=\alph*)]
\item \textit{0-10 Punkte | Begründung: }
\item \textit{0-5 Punkte | Begründung: }
\end{enumerate}

\subsection{Code-Dokumentation}
\textit{Bewertung: 0-10 Punkte}\\
\begin{enumerate}[label=\alph*)]
\item \textit{0-8 Punkte | Begründung: }
\item \textit{0-2 Punkte | Begründung: }
\end{enumerate}

\subsection{Code-Qualität}
\textit{Bewertung: 0-15 Punkte}\\
\begin{enumerate}[label=\alph*)]
\item \textit{0-5 Punkte | Begründung: }
\item \textit{0-2 Punkte | Begründung: }
\item \textit{0-3 Punkte | Begründung: }
\item \textit{0-5 Punkte | Begründung: }
\end{enumerate}
%%------------------------------------------------------

%%------------------------------------------------------
\section{Datensatz und Evaluierung}
\textit{Bewertung Datensatz und Evaluierung Gesamteindruck: 0-5 Punkte\\ 
Begründung: }
\subsection{Datensatz}
\textit{Bewertung: 0-6 Punkte}\\
\begin{enumerate}[label=\alph*)]
\item \textit{0-2 Punkte | Begründung: }
\item \textit{0-2 Punkte | Begründung: }
\item \textit{0-2 Punkte | Begründung: }
\end{enumerate}

\subsection{Reproduzierbarkeit der Ergebnisse}
\textit{Bewertung: 0-5 Punkte}\\
\begin{enumerate}[label=\alph*)]
\item \textit{0-5 Punkte | Begründung: }
\end{enumerate}

\subsection{Neue Daten}
\textit{Bewertung: 0-4 Punkte}\\
\begin{enumerate}[label=\alph*)]
\item  \textit{0-2 Punkte | Begründung: }
\item \textit{0-2 Punkte | Begründung: }
\end{enumerate}
%%------------------------------------------------------

%%------------------------------------------------------
\section{Zusammenfassung}
\label{sec:summary}
\textit{Gesamtbewertung (Summe aller Punkte): 0-120 Punkte}\\
\\
\textit{abschließende Bemerkungen, Kommentare, Verbesserungsvorschläge, etc. }
%%------------------------------------------------------


\end{document}
